\chapter{Previsão}

\begin{enumerate}
	\item[\fbox{4.1}]
	
	Fórmula 4.3.6: 
	
	\begin{align*}
		\boldsymbol{\alpha}^{(m)}{'}&\equiv
		\begin{bmatrix}
\alpha_0^{(m)}&\alpha_1^{(m)}&\alpha_2^{(m)}&\cdots&\alpha_m^{(m)}
		\end{bmatrix}\\
		&=\begin{bmatrix}
		\mu&\gamma_1+\mu^2&\gamma_2+\mu^2&\cdots&\gamma_m+\mu^2
		\end{bmatrix}\\
		&\times\begin{bmatrix}
		1&\mu&\mu&\cdots&\mu\\[0.3cm]
		\mu&\gamma_0+\mu^2&\gamma_1+\mu^2&\cdots&\gamma_{m-1}+\mu^2\\[0.3cm]
		\mu&\gamma_1+\mu^2&\gamma_0+\mu^2&\cdots&\gamma_{m-2}+\mu^2\\[0.3cm]
		\vdots&\vdots&\vdots&\ddots&\vdots\\[0.3cm]
		\mu&\gamma_{m-1}+\mu^2&\gamma_{m-2}+\mu^2&\cdots&\gamma_0+\mu^2
		\end{bmatrix}^{-1}\\
		\\
		&\equiv  {E}\big[Y_{t+1}\mathbf{X}_t'\big] {E}\big[\mathbf{X}_t\mathbf{X}_t'\big]^{-1}\\
		\text{Para }\mathbf{X}_t=\begin{bmatrix}
		1&Y_t
		\end{bmatrix}'\\
		\boldsymbol{\alpha}^{(1)}{'}&=\begin{bmatrix}
		 {E}[Y_{t+1}]&
		 {E}[Y_{t+1}Y_t]
		\end{bmatrix}
		\times
		\begin{bmatrix}
		1&E[Y_{t}]\\[0.3cm]
		E[Y_{t}]&E[Y_{t}^2]
		\end{bmatrix}^{-1}\\
		&=\begin{bmatrix}
		\mu&
		\gamma_1+\mu^2
		\end{bmatrix}
		\times
		\frac{1}{\gamma_0}
		\begin{bmatrix}
		\gamma_0+\mu^2&-\mu\\
		-\mu&1
		\end{bmatrix}\\
		&=
		\frac{1}{\gamma_0}
		\begin{bmatrix}
		\mu\gamma_0+\mu^3-\mu\gamma_1-\mu^3&-\mu^2+\gamma_1+\mu^2
		\end{bmatrix}\\
		&=\frac{1}{\gamma_0}
		\begin{bmatrix}
		\mu\gamma_0-\mu\gamma_1&\gamma_1
		\end{bmatrix}\\
		&=\begin{bmatrix}
		(1-\rho_1)\mu&\rho_1
		\end{bmatrix}
	\end{align*}
	\begin{align*}
	\hat{E}[Y_{t+1}|Y_t]&=\boldsymbol{\alpha}'\mathbf{X}_t\\
&=\begin{bmatrix}
(1-\rho_1)\mu&\rho_1
\end{bmatrix}
	\begin{bmatrix}
	1\\
	Y_t
	\end{bmatrix}\\
	\hat {E}\big[Y_{t+1}|Y_t\big]&=(1-\rho_1)\mu+\rho_1Y_t
	\end{align*}
	Portanto a previsão de $Y_{t+1}$ com base em informações somente do período $t$ é uma média ponderada entre o valor esperado de $Y_t$ ($\mu$) e $Y_t$, em que o fator de ponderação é o coeficiente de autocorrelação de primeira defasagem, $\rho_1$.
	
	\begin{enumerate}
		\item %a 
		Processo $AR(1)$: $$Y_t=c+\phi Y_{t-1}+\varepsilon_t$$
		
		Equação 4.2.9:
		$$\hat{E}[Y_{t+s}|\varepsilon_t, \varepsilon_{t-1},\cdots
		]=\mu+\frac{\psi(L)}{L^s}\varepsilon_t$$
		
		Se o processo $AR(1)$ for invertível ele pode ser escrito da forma 
		\begin{align*}
		Y_t&=\frac{c}{1-\phi L}+\frac{\varepsilon_t}{1-\phi L}\\
		&=\mu+\psi(L)\varepsilon_t, \;\;\;\; \psi(L)=1+\phi L+\phi^2L^2+\cdots
		\end{align*}
	
	Para $Y_{t+1}$ o processo se torna: $$Y_{t+1}=\mu+\varepsilon_{t+1}+\phi\varepsilon_{t}+\phi^2\varepsilon_{t-1}+\cdots$$
	O preditor linear ótimo tem a forma:
	$$\hat{E}[Y_{t+1}|\varepsilon_t, \varepsilon_{t-1},\cdots]=\mu+\phi\varepsilon_t+\phi^2\varepsilon_{t-1}+\cdots$$
	Dividindo  $\psi(L)$ por $L^s$ com $s=1$
	$$\frac{\psi(L)}{L}=L^{-1}+\phi+\phi^2L+\cdots$$
	 e tratando os $L$ com expoente negativo como zero para derivar o operador de aniquilação, o preditor ótimo para um período à frente pode ser escrito em notação de operador de defasagem:
	$$\hat{E}[Y_{t+1}|\varepsilon_t, \varepsilon_{t-1},\cdots]=\mu+\Bigg[\frac{\psi(L)}{L}\Bigg]_+\varepsilon_t$$
	Que é idêntico à equação 4.2.9 com $s=1$.
	
	
	\item %b
	Processo $MA(1)$:
	$$Y_n=\mu+\varepsilon_n+\theta\varepsilon_{n-1}$$
	
	Equação 4.5.20:
	$$\hat{E}(Y_n|Y_{n-1},Y_{n-2},\cdots,Y_1)=\mu+\frac{\theta[1+\theta^2+\theta^4+\cdots+\theta^{2(n-2)}]}{[1+\theta^2+\theta^4+\cdots+\theta^{2(n-1)}]}\bigg[Y_{n-1}-\hat{E}(Y_{n-1}|Y_{n-2},Y_{n-3},\cdots,Y_1)\bigg]$$
	
	O processo $MA(1)$ pode ser escrito como:
	
	
	
		$$Y_n-\mu=\varepsilon_n+\theta\varepsilon_{n-1}$$
	
	As autocovariâncias são:
	
	\begin{align*}
		\gamma_0&=E[(Y_n-\mu)^2]=E(\varepsilon_n+\theta\varepsilon_{n-1})(\varepsilon_n+\theta\varepsilon_{n-1})=E(\varepsilon_n^2+2\theta\varepsilon_n\varepsilon_{n-1}+\theta^2\varepsilon_{n-1}^2)=\sigma^2[{1+\theta^2}]\\
		\gamma_1&=E[(Y_n-\mu)(Y_{n-1}\-\mu)]=E[(\varepsilon_n+\theta\varepsilon_{n-1})(\varepsilon_{n-1}+\theta\varepsilon_{n-2})]\\
		&=E(\varepsilon_n \varepsilon_{n-1}+\theta\varepsilon_n\varepsilon_{n-2}+\theta\varepsilon_{n-1}^2+\theta^2\varepsilon_{n-1}\varepsilon_{n-2})=\sigma^2\theta\\
		\gamma_j&=0 \;\;\; j= 2,3,\cdots.
	\end{align*}
	
	Temos que para $n=2$
	\begin{align*}
	\hat{E}(Y_2|Y_1)&=\mu+\text{Cov}(Y_2,Y_1)\text{Var}(Y_2^2)^{-1}(Y_1-\hat{E}(Y_1))\\
	&=\mu+\frac{\theta}{1+\theta^2}(Y_1-\hat{E}(Y_1))
	\end{align*}
	Que é idêntico à equação 4.5.20 com $n=2$.

	\item %c
	 Dado que $$\hat {E}\big[Y_{t+1}|Y_t\big]=(1-\rho_1)\mu+\rho_1Y_t$$
	 Podemos rearranjar a expressão e obter 
	 $$\hat {E}\big[Y_{t+1}|Y_t\big]=\mu+\rho_1(Y_t-\mu)$$
	 Como o coeficiente de correlação de primeira defasagem de um processo $AR(2)$ é
	 
	 $$\rho_1=\frac{\phi_1}{1-\phi_2}$$
	 
	 O preditor se torna
	 $$\hat {E}\big[Y_{t+1}|Y_t\big]=\mu+\frac{\phi_1}{1-\phi_2}(Y_t-\mu)$$
	 
	 O resíduo desta predição é 
	 
	 \begin{align*}
	 \bar{Y}_{t+1|t}&=Y_{t+1}-\Bigg[\mu+\frac{\phi_1}{1-\phi_2}(Y_t-\mu)\Bigg]\\
	 &=\mu+\phi_1Y_t+\phi_2Y_{t-1}+\varepsilon_{t+1}-\mu-\frac{\phi_1}{1-\phi_2}(Y_t-\mu)\\
	 &=\frac{\phi_1}{1-\phi_2}\mu-\Bigg(\frac{\phi_1\phi_2}{1-\phi_2}\Bigg)Y_t+\phi_2Y_{t-1}+\varepsilon_{t+1}
	 \end{align*}
	
	Verificando correlação entre resíduo e valores defasados: \begin{align*}
			E(\bar{Y}_{t+1|t}Y_t)&=E\Bigg\{\Bigg[\frac{\phi_1}{1-\phi_2}\mu-\Bigg(\frac{\phi_1\phi_2}{1-\phi_2}\Bigg)Y_t+\phi_2Y_{t-1}+\varepsilon_{t+1}\Bigg]\Bigg[\mu+\phi_1Y_{t-1}+\phi_2Y_{t-2}+\varepsilon_t\Bigg]\Bigg\}\\
			&\neq0
			\end{align*}
	Então, $\hat{Y}_{t+1|t}$ e $Y_t$ são correlacionados.
	
	\begin{align*}
		E(\bar{Y}_{t+1|t}Y_{-1})&=E\Bigg\{\Bigg[\frac{\phi_1}{1-\phi_2}\mu-\Bigg(\frac{\phi_1\phi_2}{1-\phi_2}\Bigg)Y_t+\phi_2Y_{t-1}+\varepsilon_{t+1}\Bigg]\Bigg[\mu+\phi_1Y_{t-2}+\phi_2Y_{t-3}+\varepsilon_{t-1}\Bigg]\Bigg\}\\
		&\neq0
	\end{align*}
	Então, $\hat{Y}_{t+1|t}$ e $Y_{t-1}$ também são correlacionados.
	
	\end{enumerate}
	
	 
	
	\item[\fbox{4.2}]
		
		Como $s=q=1$, temos um processo $MA(1)$:
		
		$$Y_t=\mu+\varepsilon_t+\theta\varepsilon_{t-1}$$
		
		E a predição a ser realizada para um período à frente:
		
		$$E[Y_{t+1}|Y_t].$$
		
		Dado que o processo $MA(1)$ possa ser escrito na forma 
		$$Y_t-\mu=(1+\theta L)\varepsilon_t.$$
		
		Sendo a fórmula de predição de \emph{Wiener-Kolmogorov} para $s=1$:
		$$\hat{Y}_{t+1|t}=\mu+\Bigg[\frac{\psi(L)}{L}\Bigg]_+\varepsilon_t$$
		Substituindo o processo $$MA(1)$$ na fórmula temos:
		
		$$\hat{Y}_{t+1|t}=\mu+\Bigg[\frac{1+\theta L}{L}\Bigg]_+\frac{(Y_t-\mu)}{1+\theta L}$$
		
		Para o processo $MA(1)$:
		
		$$\Big[\frac{1+\theta L}{L}\Big]_+=\theta$$

		Então podemos escrever a fórmula de \emph{Wiener-Kolmogorov} como
		
		$$\hat{Y}_{t+1|t}=\mu+\theta\frac{(Y_t-\mu)}{1+\theta L}$$
		
		Expandindo a soma infinita representada:
		
		$$\hat{Y}_{t+1|t}=\mu+\theta(Y_t-\mu)-\theta^2(Y_{t-1}-\mu)+\theta^3(Y_{t-2}-\mu)-\cdots+(-1)^{m-1}\theta^{m}(Y_{t-m+1}-\mu)$$
		
		Equivalente à equação 4.3.3.
		
	 
	
	\item[\fbox{4.3}]
	
\begin{align*}
	\boldsymbol{\Omega}&=\begin{bmatrix*}[r]
		1&-2&3\\
		-2&6&-4\\
		3&-4&12
	\end{bmatrix*}\\
	\mathbf{E_1}&=\begin{bmatrix*}[r]
	1&0&0\\
	2&1&0\\
	-3&0&1
	\end{bmatrix*}\\
	\mathbf{E}_1\boldsymbol{\Omega}\mathbf{E}_1{'}=\mathbf{H}&=\begin{bmatrix*}[r]
	1&0&0\\
	2&1&0\\
	-3&0&1
	\end{bmatrix*}\begin{bmatrix*}[r]
	1&-2&3\\
	-2&6&-4\\
	3&-4&12
	\end{bmatrix*}\begin{bmatrix*}[r]
	1&2&-3\\
	0&1&0\\
	0&0&1
	\end{bmatrix*}\\
	&=\begin{bmatrix*}[r]
	1&-2&3\\
	0&2&2\\
	0&2&3
	\end{bmatrix*}
	\begin{bmatrix*}[r]
	1&2&-3\\
	0&1&0\\
	0&0&1
	\end{bmatrix*}\\
	&=\begin{bmatrix}
		1&0&0\\
		0&2&2\\
		0&2&3
	\end{bmatrix}
\end{align*}	

\begin{align*}
	\mathbf{E}_2&=\begin{bmatrix*}[r]
	1&0&0\\
	0&1&0\\
	0&-1&1
	\end{bmatrix*}\\
	\mathbf{E}_2\mathbf{H}\mathbf{E}_2{'}&=\begin{bmatrix*}[r]
	1&0&0\\
	0&1&0\\
	0&-1&1
	\end{bmatrix*}
	\begin{bmatrix}
	1&0&0\\
	0&2&2\\
	0&2&3
	\end{bmatrix}
	\begin{bmatrix*}[r]
	1&0&0\\
	0&1&-1\\
	0&0&1
	\end{bmatrix*}\\
	&=\begin{bmatrix}
	1&0&0\\
	0&2&2\\
	0&0&1
	\end{bmatrix}
		\begin{bmatrix*}[r]
	1&0&0\\
	0&1&-1\\
	0&0&1
	\end{bmatrix*}\\
	&=\begin{bmatrix}
	1&0&0\\
	0&2&0\\
	0&0&1
	\end{bmatrix}
\end{align*}
\begin{align*}
\mathbf{E}_2\mathbf{E}_1\boldsymbol{\Omega}\mathbf{E}_1{'}\mathbf{E}_2{'}=\mathbf{E}_2\mathbf{E}_1\boldsymbol{\Omega}(\mathbf{E}_2\mathbf{E}_1)'&=\mathbf{A}^{-1}\boldsymbol{\Omega}\mathbf{A}{'}^{-1}=\mathbf{D}\\
\boldsymbol{\Omega}&=\mathbf{A}\mathbf{D}\mathbf{A}'\\
\mathbf{A}=(\mathbf{E}_2\mathbf{E}_1)^{-1}&=
\begin{bmatrix}\begin{bmatrix}
1&0&0\\
0&1&0\\
0&-1&1
\end{bmatrix}&
\begin{bmatrix*}[r]
1&0&0\\
2&1&0\\
-3&0&1
\end{bmatrix*}\end{bmatrix}^{-1}\\
&=\begin{bmatrix*}[r]
1&0&0\\
2&1&0\\
-5&-1&1
\end{bmatrix*}^{-1}\\
&=\begin{bmatrix*}[r]
	1&0&0\\
	-2&1&0\\
	3&1&1
\end{bmatrix*}
\end{align*}
$$\mathbf{A}\boldsymbol{\Omega}\mathbf{A}'=\begin{bmatrix*}[r]
1&0&0\\
-2&1&0\\
3&1&1
\end{bmatrix*}\\
\begin{bmatrix*}[r]
1&0&0\\
0&2&0\\
0&0&1
\end{bmatrix*}
\begin{bmatrix*}[r]
1&-2&3\\
0&1&1\\
0&0&1
\end{bmatrix*}$$


	 

	
	\item[\fbox{4.4}]
	
	Não pois o coeficiente de $Y_2$ da projeção linear de $Y_4$ contra $Y_3$, $Y_2$ e $Y_1$ é:
	
	Dado que
	
	$$Y_4=\bar{Y}_4+k_{43}k_{33}^{-1}\bar{Y}_3+h_{41}h_{22}^{-1}\bar{Y}_2+\Omega_{41}\Omega_{11}^{-1}\bar{Y}_1,$$
	
	temos ainda que:
	  $$\bar{Y}_1=Y,$$
	 $$\bar{Y}_2=Y_2-\Omega_{21}\Omega_{11}^{-1}\bar{Y}_1$$
	 $$\bar{Y}_3=Y_3-\Omega_{31}\Omega_{11}^{-1}Y_1-h_{32}h_{22}^{-1}(Y_2-\Omega_{21}\Omega_{11}^{-1}\bar{Y}_1)$$
	 
	 Substituindo em $Y_4$ nos dá:
	 	\begin{align*}
	 	Y_4&=\bar{Y}_4+k_{43}k_{33}^{-1}(Y_3-\Omega_{31}\Omega_{11}^{-1}Y_1-h_{32}h_{22}^{-1}(Y_2-\Omega_{21}\Omega_{11}^{-1}{Y}_1))\\
	 	&+h_{41}h_{22}^{-1}(Y_2-\Omega_{21}\Omega_{11}^{-1}){Y}_1+\Omega_{41}\Omega_{11}^{-1}{Y}_1
	 	\end{align*}
	 Podemos observar que o coeficiente de $Y_2$ é 
	 $$h_{41}h_{22}^{-1}-k_{43}k_{33}^{-1}h_{32}h_{22}^{-1}$$
	 
	 Já o elemento $(4,2)$ de $\mathbf{A}$ é 
	 
	 $$h_{41}h_{22}^{-1}$$
	 
	 Portanto a resposta é não.
	 
	 \item[\fbox{4.5}]
	 
	 
	 
\end{enumerate}