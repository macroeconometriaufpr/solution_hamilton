\chapter{Análise Spectral}

\begin{enumerate}
	\item[6.1]
	Equação 6.1.12: $$s_Y(\omega)=(2\pi)^{-1}\sigma^2[1+\theta^2+2\theta\cos(\omega)].$$
	equação 6.1.6: $$s_Y(\omega)=\frac{1}{2\pi}\bigg\{\gamma_0+2\sum\limits_{j=1}^{\infty}\gamma_j\cos(\omega j)\bigg\}.$$
	Autocovariâncias de um processo $MA(1)$:
	\begin{align*}
	\gamma_0&=\sigma^2[1+\theta^2]\\
	\gamma_1&=\sigma^2\theta\\
	\gamma_k&=0\quad k\geqslant2
	\end{align*}

Substituindo as autocovariâncias na equação 6.1.6, temos:

\begin{align*}
	s_Y(\omega)&=\frac{1}{2\pi}\bigg\{\sigma^2[1+\theta^2]+2\sigma^2\theta\cos(\omega )\bigg\}\\
	&={2\pi}^{-1}\sigma^2[1+\theta^2+2\theta\cos(\omega )]\quad_{\blacksquare}
\end{align*}

	\item[6.2]
	Equação 6.1.9:
	$$s_Y(\omega)=\frac{\sigma^2}{2\pi}$$
	Equação 6.1.12:
	$$s_Y(\omega)=(2\pi)^{-1}\sigma^2[1+\theta^2+2\theta\cos(\omega)].$$
	Equação 6.1.17:
	$$\int_{-\pi}^{\pi}s_Y(\omega)\mathrm{d}\omega=\gamma_0$$

	Integrando 6.1.9:
	\begin{align*}
		\int_{-\pi}^{\pi}\frac{\sigma^2}{2\pi}\mathrm{d}\omega&=\frac{\sigma^2}{2\pi}\bigg[\omega\bigg]^{\pi}_{-\pi}\\
		&=\frac{\sigma^2}{2\pi}[\pi-(-\pi)]\\
		&=\frac{\sigma^2}{2\pi}2\pi\\
		&=\sigma^2
	\end{align*}

	Integrando 6.1.12:
	\begin{align*}
		\int_{-\pi}^{\pi}(2\pi)^{-1}\sigma^2[1+\theta^2+2\theta\cos(\omega)]\mathrm{d}\omega&=(2\pi)^{-1}\sigma^2\bigg[\omega+\theta^2\omega+2\theta\mathrm{sen}(\omega)\bigg]_{-\pi}^{\pi}\\
		&=(2\pi)^{-1}\sigma^2\bigg[(\pi+\theta^2\pi+2\theta\mathrm{sen}(\pi))-(-\pi+\theta^2(-\pi)+2\theta\mathrm{sen}(-\pi))\bigg]\\
		&=(2\pi)^{-1}\sigma^2\bigg[(\pi+\theta^2\pi+2\theta\mathrm{sen}(\pi))+\pi+\theta^2\pi-2\theta\mathrm{sen}(-\pi))\bigg]\\
		&=(2\pi)^{-1}\sigma^2\bigg[2\pi(1+\theta^2)\bigg]\\
		&=\sigma^2[1+\theta^2]\quad_{\blacksquare}
	\end{align*}
Já que $\mathrm{sen}(\pi)=\mathrm{sen}(-\pi)=0$.
\end{enumerate}
