\setcounter{chapter}{9}
\chapter{Processos Vetoriais Estacionários em Covariância}

\begin{enumerate}
	\item[\fbox{10.1}]
	
	Equação 10.2.19: $$\text{vec}\begin{bmatrix}
	\Gamma_0&\Gamma_1&\cdots&\Gamma_{p-1}\\
	\Gamma_1'&\Gamma_0&\cdots&\Gamma_{p-2}\\
	\vdots&\vdots&\vdots&\vdots\\
	\Gamma_{p-1}'&\Gamma_{p-2}'&\cdots&\Gamma_0
	\end{bmatrix}=[\mathbf{I}_{r^2}-\mathbf{\mathscr{A}}]^{-1}\text{vec}(\mathbf{Q}).$$
	Sendo que $r=np$ e $$\mathbf{\mathscr{A}}\equiv \mathbf{F}_{(r\times r)}\otimes\mathbf{F}_{(r\times r)}$$
	
	
	Um processo escalar ($n=1$) $AR(p)$ tem o forma:
	
	\begin{align*}
		y_t &= \phi_1y_{t-1}+\phi_2y_{t-2}+\cdots+\phi_py_{t-p}+\varepsilon_t\\
	\end{align*}

Ou em forma vetorial:
	\begin{align*}
		\boldsymbol{\xi}_t = \mathbf{F}\boldsymbol{\xi}_{t-1}+\mathbf{v}_t,
	\end{align*}

Avaliando a fórmula 10.2.19 para um processo escalar temos:

	$$\text{vec}\begin{bmatrix}
\gamma_0&\gamma_1&\cdots&\gamma_{p-1}\\
\gamma_1 &\gamma_0&\cdots&\gamma_{p-2}\\
\vdots&\vdots&\vdots&\vdots\\
\gamma_{p-1} &\gamma_{p-2} &\cdots&\gamma_0
\end{bmatrix}=[\mathbf{I}_{p^2}-\mathbf{\mathscr{A}}]^{-1}\text{vec}(\mathbf{Q}).$$

Com $$\mathbf{\mathscr{A}}\equiv \mathbf{F}_{(p\times p)}\otimes\mathbf{F}_{(p\times p)}$$

Temos ainda que $\mathbf{Q}$ para um processo escalar se torna:

\begin{align*}\mathbf{Q}=E[\mathbf{vv}']&=E\begin{Bmatrix}
\begin{bmatrix}
\varepsilon_t\\
0\\
\vdots\\
0
\end{bmatrix}&
\begin{bmatrix}
\varepsilon_t&0&\cdots&0
\end{bmatrix}\\
\end{Bmatrix}\\
&=E\begin{bmatrix}
	\varepsilon_t^2&0&\cdots&0\\
	0&0&\cdots&0\\
	\vdots&\vdots&\vdots&\vdots\\
	0&0&\cdots&0
\end{bmatrix}=\sigma^2\begin{bmatrix}
1&0&\cdots&0\\
0&0&\cdots&0\\
\vdots&\vdots&\vdots&\vdots\\
0&0&\cdots&0
\end{bmatrix}
	\end{align*}	
	Realizando a operação vec nos dois lados da equação  10.2.19 para o processo escalar:
	$$\begin{bmatrix}
	\gamma_0\\
	\gamma_1\\
	\vdots\\
	\gamma_{p-1}\\
	\gamma_1\\
	\gamma_0\\
	\vdots\\
	\gamma_{p-2}\\
	\vdots\\
	\gamma_{p-1}\\
	\gamma_{p-2}\\
	\vdots\\
	\gamma_0
	\end{bmatrix}=[\mathbf{I}_{p^2}-\mathbf{\mathscr{A}}]^{-1}\sigma^2\begin{bmatrix}
	1\\
	0\\
	\vdots\\
	0\\
	0\\
	0\\
	\vdots\\
	0\\
	\vdots\\
	0\\
	0\\
	\vdots\\
	0
	\end{bmatrix}$$
	
	dados que $[\mathbf{I}_{p^2}-\mathbf{\mathscr{A}}]^{-1}$ é uma matriz $p^2\times p^2$ e o vetor coluna do lado direito tem dimensão $P^2 \times 1$, o produto dos dois termos pode ser feito da seguinte forma. Deixe que $[\mathbf{I}_{p^2}-\mathbf{\mathscr{A}}]^{-1}$ seja igual a uma matriz $\mathbf{H}$  e que ela seja particionada em $p^2$ vetores coluna $\mathbf{h}_j$ de dimensão $p^2\times 1$ com $j=1,2,\cdots,p^2$.
	
	$$\mathbf{H}=\begin{bmatrix}
	\mathbf{h}_1&\mathbf{h}_2&\cdots&\mathbf{h}_{p^2}
	\end{bmatrix}$$
	
	Então o produto fica um vetor coluna de dimensão $p^2 \times 1$:
	$$\begin{bmatrix}
	\mathbf{h}_1&\mathbf{h}_2&\cdots&\mathbf{h}_{p^2}
	\end{bmatrix}\begin{bmatrix}
	1\\
	0\\
	\vdots\\
	o
	\end{bmatrix}_{(p^2\times 1)}=\begin{bmatrix}
	\mathbf{h}_1(1)+\mathbf{h}_2(0)+\cdots+\mathbf{h}_{p^2}(0)
	\end{bmatrix}_{(p^2\times 1)}=\begin{bmatrix}
	\mathbf{h}_1
	\end{bmatrix}_{(p^2\times 1)}$$
	
	Com isso os $p$ primeiros termos do lado esquerdo do processo escalar empilhado é o vetor coluna $$\begin{bmatrix}
	\gamma_0\\
	\gamma_1\\
	\gamma_2\\
	\vdots\\
	\gamma_{p-1}
	\end{bmatrix}$$
 e os $p$ primeiros termos do lado direito desta mesma equação é um vetor composto pelos $p$ primeiros elementos do vetor $$\sigma^2\begin{bmatrix}
 \mathbf{h}_1
 \end{bmatrix}_{(p^2\times 1)},$$
 que é equivalente aos $p$ primeiros elementos da primeira coluna de $$\sigma^2\mathbf{H}\equiv\sigma^2[\mathbf{I}_{p^2}-\mathbf{\mathscr{A}}]^{-1}\quad_{\blacksquare}$$
 
 \item[\fbox{10.2}]
 
	 \begin{enumerate}
	 	\item %10.2a
	 	Sendo $\mathbf{y}=(X_t,Y_t)'$:
	 	\begin{align*}
	 		\boldsymbol{\Gamma}_k&=E[(\mathbf{y}_t-\boldsymbol{\mu})(\mathbf{y}_{t-k}-\boldsymbol{\mu})']
	 		\end{align*}
	 
 Dados que no exercício $\mu_X=\mu_Y=0$:
	 \begin{align*}
	 	\boldsymbol{\Gamma}_k&=E[(\mathbf{y}_t)(\mathbf{y}_{t-k})']\\
	 	&=E\begin{Bmatrix}
	 		\begin{bmatrix}
	 			X_t\\
	 			Y_t
	 		\end{bmatrix}&\begin{bmatrix}
	 		X_{t-k}&Y_{t-k}
 		\end{bmatrix}
	 	\end{Bmatrix}\\
 	&=E\begin{bmatrix}
 		X_tX_{t-k}&X_tY_{t-k}\\
 		Y_tX_{t-k}&Y_tY_{t-k}
 	\end{bmatrix}
	 \end{align*}
 
	 \begin{align*}
	 	\boldsymbol{\Gamma}_0&=E\begin{bmatrix}
	 		X_tX_t&X_tY_t\\
	 		Y_tX_t&Y_tY_t
	 	\end{bmatrix}\\
 	&=E\begin{bmatrix}
 		(\varepsilon_t+\theta\varepsilon_{t-1})(\varepsilon_t+\theta\varepsilon_{t-1})&(\varepsilon_t+\theta\varepsilon_{t-1})(h_1X_{t-1}+u_t)\\
 		(h_1X_{t-1}+u_t)(\varepsilon_t+\theta\varepsilon_{t-1})&(h_1X_{t-1}+u_t)(h_1X_{t-1}+u_t)
 	\end{bmatrix}\\
 &=E\begin{bmatrix}
 	(\varepsilon_t+\theta\varepsilon_{t-1})(\varepsilon_t+\theta\varepsilon_{t-1})&(\varepsilon_t+\theta\varepsilon_{t-1})(h_1(\varepsilon_{t-1}+\theta\varepsilon_{t-2})+u_t)\\
 	(h_1(\varepsilon_{t-1}+\theta\varepsilon_{t-2})+u_t)(\varepsilon_t+\theta\varepsilon_{t-1})&(h_1(\varepsilon_{t-1}+\theta\varepsilon_{t-2})+u_t)(h_1(\varepsilon_{t-1}+\theta\varepsilon_{t-2})+u_t)
 \end{bmatrix}\\
		&=\begin{bmatrix}
			\sigma^2_{\varepsilon}[1+\theta^2]&h_1\theta\sigma^2_{\varepsilon}\\
			h_1\theta\sigma^2_{\varepsilon}&\sigma^2_{\varepsilon}h_1^2[1+\theta^2]+\sigma^2_u
		\end{bmatrix}
	 \end{align*}
 
	 \begin{align*}
	 	\boldsymbol{\Gamma}_1&=E\begin{bmatrix}
	 		X_tX_{t-1}&X_tY_{t-1}\\
	 		Y_tX_{t-1}&Y_tY_{t-1}
	 	\end{bmatrix}\\
	 	&=E\begin{bmatrix}
	 		(\varepsilon_t+\theta\varepsilon_{t-1})(\varepsilon_{t-1}+\theta\varepsilon_{t-2})&(\varepsilon_t+\theta\varepsilon_{t-1})(h_1X_{t-2}+u_{t-1})\\
	 		(h_1X_{t-1}+u_t)(\varepsilon_{t-1}+\theta\varepsilon_{t-2})&(h_1X_{t-1}+u_t)(h_1X_{t-2}+u_{t-1})
	 	\end{bmatrix}\\
 	&=E\begin{bmatrix}
 		(\varepsilon_t+\theta\varepsilon_{t-1})(\varepsilon_{t-1}+\theta\varepsilon_{t-2})&(\varepsilon_t+\theta\varepsilon_{t-1})(h_1(\varepsilon_{t-2}+\theta\varepsilon_{t-3})+u_{t-1})\\
 		(h_1(\varepsilon_{t-1}+\theta\varepsilon_{t-2})+u_t)(\varepsilon_{t-1}+\theta\varepsilon_{t-2})&(h_1(\varepsilon_{t-1}+\theta\varepsilon_{t-2})+u_t)(h_1(\varepsilon_{t-2}+\theta\varepsilon_{t-3})+u_{t-1})
 	\end{bmatrix}\\
 &=\begin{bmatrix}	
 	\theta\sigma^2_{\varepsilon} &0\\
 	h_1\sigma^2_{\varepsilon}[1+\theta^2]&h_1^2\theta\sigma^2_{\varepsilon}
 \end{bmatrix}\\
	\\
	\boldsymbol{\Gamma}_2&=E\begin{bmatrix}
		X_tX_{t-2}&X_tY_{t-2}\\
		Y_tX_{t-2}&Y_tY_{t-2}
	\end{bmatrix}\\
	&=E\begin{bmatrix}
		(\varepsilon_t+\theta\varepsilon_{t-1})(\varepsilon_{t-2}+\theta\varepsilon_{t-3})&(\varepsilon_t+\theta\varepsilon_{t-1})(h_1X_{t-3}+u_{t-2})\\
		(h_1X_{t-1}+u_t)(\varepsilon_{t-2}+\theta\varepsilon_{t-3})&(h_1X_{t-1}+u_t)(h_1X_{t-3}+u_{t-2})
	\end{bmatrix}\\
		&=E\begin{bmatrix}
		(\varepsilon_t+\theta\varepsilon_{t-1})(\varepsilon_{t-2}+\theta\varepsilon_{t-3})&(\varepsilon_t+\theta\varepsilon_{t-1})(h_1(\varepsilon_{t-3}+\theta\varepsilon_{t-4})+u_{t-2})\\
		(h_1(\varepsilon_{t-1}+\theta\varepsilon_{t-2})+u_t)(\varepsilon_{t-2}+\theta\varepsilon_{t-3})&(h_1(\varepsilon_{t-1}+\theta\varepsilon_{t-2})+u_t)(h_1(\varepsilon_{t-3}+\theta\varepsilon_{t-4})+u_{t-2})
	\end{bmatrix}\\
	&=\begin{bmatrix}
		0&0\\
		h_1\theta\sigma^2_{\varepsilon}&0
	\end{bmatrix}
 	\end{align*}
 
  Temos ainda que $\boldsymbol{\Gamma}_j=\mathbf{0}$ para $j\geqslant3$.$\quad_\blacksquare$
  
  
  \item %10.2b
  
  Equação 10.4.3: $$\mathbf{s}_{\mathbf{y}}(\omega)=(2\pi)^{-1}\mathbf{G}_{\mathbf{y}}(e^{-i\omega})=(2\pi)^{-1}\sum\limits_{k=-\infty}^{\infty}\mathbf{\Gamma}_ke^{-i\omega k}.$$
  Neste caso $k=-1,0,1$. Lembrando que $\boldsymbol{\Gamma}_{-k}=\boldsymbol{\Gamma}_{k}'$:
  
  %\begin{landscape}
  	
  	
  	
  \begin{align*}\mathbf{s}_{\mathbf{y}}(\omega)=(2\pi)^{-1}\sum\limits_{k=-1}^1\boldsymbol{\Gamma}_ke^{-i\omega k}  
  \end{align*}
		Que de acordo com 10.4.11 fica:
		\begin{align*}
			\mathbf{s}_{\mathbf{y}}(\omega)&=\frac{1}{2\pi}\begin{bmatrix}
				\sum\limits_{k=-2}^2\gamma_{XX}^{(k)}\cos(\omega k)&\sum\limits_{k=-2}^2\gamma_{XY}^{(k)}\{\cos(\omega k)-i\,\text{sen}(\omega k)\}\\
				\sum\limits_{k=-2}^2\gamma_{YX}^{(k)}\{\cos(\omega k)-i\,\text{sen}(\omega k)\}&\sum\limits_{k=-2}^2\gamma_{YY}^{(k)}\cos(\omega k)
			\end{bmatrix}
		\end{align*}
	Analisando cada termo separadamente, lembrando que $\cos(0)=1$ e $\cos(\omega)=\cos(-\omega)$:
	\begin{align*}
	\mathbf{s}_{\mathbf{y}}(\omega)^{11}=	\sum\limits_{k=-2}^2\gamma_{XX}^{(k)}\cos(\omega k)&=\theta\sigma^2_{\varepsilon}\cos(-\omega)+\sigma^2_{\varepsilon}[1+\theta^2]+\theta\sigma^2_{\varepsilon}\cos(\omega)\\
		&=\sigma^2_{\varepsilon}[1+2\theta\cos(\omega)+\theta^2]\\
		\\
		\mathbf{s}_{\mathbf{y}}(\omega)^{22}=\sum\limits_{k=-2}^2\gamma_{YY}^{(k)}\cos(\omega k)&=h_1^2\theta\sigma^2_{\varepsilon}\cos(-\omega)+\sigma^2_{\varepsilon}h_1^2[1+\theta^2]\\
		&+\sigma^2_u+h_1^2\theta\sigma^2_{\varepsilon}\cos(\omega)\\
		&=\sigma^2_{\varepsilon}[h_1^2[1+\theta^2]+h_1^22\theta\cos(\omega)]+\sigma^2_u\\
		\\
		\mathbf{s}_{\mathbf{y}}(\omega)^{12}=\sum\limits_{k=-2}^2\gamma_{XY}^{(k)}e^{-i\omega k}&=h_1\theta\sigma^2_{\varepsilon}e^{i2\omega}
		+h_1\sigma^2_{\varepsilon}[1+\theta^2]e^{i\omega}+h_1\theta\sigma^2_{\varepsilon}\\
		\\
		\mathbf{s}_{\mathbf{y}}(\omega)^{21}=\sum\limits_{k=-2}^2\gamma_{YX}^{(k)}e^{-i\omega k}&=h_1\theta\sigma^2_{\varepsilon}+h_1\sigma^2_{\varepsilon}[1+\theta^2]e^{-i\omega}+h_1\theta\sigma^2_{\varepsilon}e^{-i2\omega}\\
	\end{align*}
	O \emph{spectrum} populacional fica:
	
	\begin{align*}
		\mathbf{s}_{\mathbf{y}}(\omega)=\frac{1}{2\pi}\begin{bmatrix*}
			\mathbf{s}_{\mathbf{y}}(\omega)^{11}&\mathbf{s}_{\mathbf{y}}(\omega)^{12}\\
			\mathbf{s}_{\mathbf{y}}(\omega)^{21}&\mathbf{s}_{\mathbf{y}}(\omega)^{22}
		\end{bmatrix*}
		\end{align*}
	
	O \emph{cospectrum} entre $X$ e $Y$ é:
	\begin{align*}
		c_{XY}(\omega)&=(2\pi)^{-1}h_1\sigma^2_{\varepsilon}(\theta\cos(2\omega)+[1+\theta^2]\cos(\omega)+\theta)
	\end{align*}

	E o \emph{quadrature spectrum} entre $X$ e $Y$ é:
	$$q_{XY}(\omega)=-(2\pi)^{-1}h_1\sigma^2_{\varepsilon}(\theta\text{sen}(2\omega)+[1+\theta^2]\text{sen}(\omega))$$
	
	%\end{landscape}
	
	\item %10.2c
	
	Equação 10.4.45:
	\begin{align*}
		\mathbf{s}_{\mathbf{y}}(\omega)&=\begin{bmatrix}
	1&0\\
	h(e^{-\omega})&1
	\end{bmatrix}\begin{bmatrix}
	s_{XX}(\omega)&0\\0&s_{UU}(\omega)
	\end{bmatrix}\begin{bmatrix}
	1&h(e^{i\omega})\\0&1
	\end{bmatrix}\\
	&=\begin{bmatrix}
	s_{XX}(\omega)&s_{XX}(\omega)h(e^{i\omega})\\
	h(e^{-i\omega})s_{XX}(\omega)&h(e^{-i\omega})s_{XX}(\omega)h(e^{i\omega})+s_{UU}(\omega)
	\end{bmatrix},
	\end{align*}
	
	onde
	\begin{align*}
		h(e^{-i\omega})=\sum\limits_{k=-\infty}^{\infty}h_ke^{-i\omega k}\\
	\end{align*}
Que neste caso  $h(e^{-i\omega})=h_1e^{-i\omega}$ pois $h_k=0$ para $k \neq 1$.
	
	\begin{align*}
			s_{XX}(\omega)&=(2\pi)^{-1}\sum\limits_{k=-\infty}^{\infty}\gamma_{XX}^{(k)}e^{-i\omega k}\\
			&=(2\pi)^{-1}\bigg[\sigma^2_{\varepsilon}\theta e^{-i\omega}+\sigma^2_{\varepsilon}[1+\theta^2]+\sigma^2_{\varepsilon}\theta e^{i\omega}\bigg]\\
			&=(2\pi)^{-1}\sigma^2_{\varepsilon}\bigg[1+2\theta\cos(\omega)+\theta^2\bigg]\\
			\\
			s_{UU}(\omega)&=(2\pi)^{-1}\sigma^2_u\\
			\\
			s_{XX}(\omega)[h_1e^{-i\omega}]&=(2\pi)^{-1}\bigg[\sigma^2_{\varepsilon}\theta e^{-i\omega}+\sigma^2_{\varepsilon}[1+\theta^2]+\sigma^2_{\varepsilon}\theta e^{i\omega}\bigg][h_1e^{-i\omega}]\\
			&=(2\pi)^{-1}h_1\sigma^2_{\varepsilon}\bigg[\theta e^{-i2\omega}+[1+\theta^2]e^{-i\omega}+\theta\bigg]\\
			&=(2\pi)^{-1}h_1\sigma^2_{\varepsilon}\bigg[\theta \{\cos(2\omega)-i\text{sen}(2\omega)\}+[1+\theta^2]\{\cos(\omega)-i\text{sen}(\omega)\}+\theta\bigg]\\
			\\
			s_{XX}(\omega)h(e^{i\omega})&=(2\pi)^{-1}\bigg[\sigma^2_{\varepsilon}\theta e^{-i\omega}+\sigma^2_{\varepsilon}[1+\theta^2]+\sigma^2_{\varepsilon}\theta e^{i\omega}\bigg][h_1e^{i\omega}]\\
			&=(2\pi)^{-1}h_1\sigma^2_{\varepsilon}\bigg[\theta +[1+\theta^2]e^{i\omega}+\theta e^{i2\omega}\bigg]\\
			&=(2\pi)^{-1}h_1\sigma^2_{\varepsilon}\bigg[\theta +[1+\theta^2]\{\cos(\omega)+i\text{sen}(\omega)\}+\theta \{\cos(2\omega)+i\text{sen}(2\omega)\}\bigg]\\
			\\
			h(e^{-i\omega})s_{XX}(\omega)h(e^{i\omega})+s_{UU}(\omega)&=(2\pi)^{-1}h_1\sigma^2_{\varepsilon}\bigg[\theta e^{-i2\omega}+[1+\theta^2]e^{-i\omega}+\theta\bigg][h_1e^{i\omega}]+s_{UU}\\
			&=(2\pi)^{-1}h_1^2\sigma^2_{\varepsilon}\bigg[\theta e^{-i\omega}+[1+\theta^2]+\theta e^{i\omega}\bigg]+(2\pi)^{-1}\sigma^2_u\\
			&=(2\pi)^{-1}h_1^2\sigma^2_{\varepsilon}\bigg[2\theta \{\cos(\omega)\}+[1+\theta^2]\bigg]+(2\pi)^{-1}\sigma^2_u
	\end{align*}
		
		\item %10.2d
		
		Equação 10.4.49:
		$$h_k=(2\pi)^{-1}\int_{-\pi}^{\pi}\frac{s_{YX}(\omega)}{s_{XX}(\omega)}e^{i\omega k} d\omega.$$
		
		Quando $k=1$:
		\begin{align*}
			h_1&=(2\pi)^{-1}\int_{-\pi}^{\pi}\frac{(2\pi)^{-1}[h_1\theta\sigma^2_{\varepsilon}+h_1\sigma^2_{\varepsilon}[1+\theta^2]e^{-i\omega}+h_1\theta\sigma^2_{\varepsilon}e^{-i2\omega}]}{(2\pi)^{-1}\sigma^2_{\varepsilon}\bigg[1+2\theta\cos(\omega)+\theta^2\bigg]}e^{i\omega} d\omega\\
			&=(2\pi)^{-1}\int_{-\pi}^{\pi}\frac{[h_1\theta\sigma^2_{\varepsilon}e^{i\omega}+h_1\sigma^2_{\varepsilon}[1+\theta^2]+h_1\theta\sigma^2_{\varepsilon}e^{-i\omega}]}{\sigma^2_{\varepsilon}\bigg[1+2\theta\cos(\omega)+\theta^2\bigg]} d\omega\\
			&=(2\pi)^{-1}\int_{-\pi}^{\pi}\frac{\bigg(h_1\sigma^2_{\varepsilon}\bigg[\theta e^{i\omega}+[1+\theta^2]+\theta e^{-i\omega}\bigg]\bigg)}{\sigma^2_{\varepsilon}\bigg[1+2\theta\cos(\omega)+\theta^2\bigg]} d\omega\\
			&=(2\pi)^{-1}\int_{-\pi}^{\pi}\frac{\bigg(h_1\sigma^2_{\varepsilon}\bigg[1+2\theta\cos(\omega)+\theta^2\bigg]\bigg)}{\sigma^2_{\varepsilon}\bigg[1+2\theta\cos(\omega)+\theta^2\bigg]} d\omega\\
			&=(2\pi)^{-1}\int_{-\pi}^{\pi}{h_1} d\omega=h_1
		\end{align*}
		
		Quando $k\neq 1$:
		
		\begin{align*}
			h_k&=(2\pi)^{-1}\int_{-\pi}^{\pi}h_1e^{-i\omega}e^{i\omega k}d\omega\\
			&=(2\pi)^{-1}\int_{-\pi}^{\pi}h_1e^{(k-1)i\omega}d\omega\\
			&=(2\pi)^{-1}\int_{-\pi}^{\pi}h_1\cos([k-1]\omega)d\omega+i(2\pi)^{-1}\int_{-\pi}^{\pi}h_1\text{sen}([k-1]\omega)d\omega\\
			&=([k-1]2\pi)^{-1}h_1\bigg[\text{sen}([k-1]\omega)\bigg]_{-\pi}^{\pi}-i([k-1]2\pi)^{-1}h_1\bigg[\text{cos}([k-1]\omega)\bigg]_{-\pi}^{\pi}\\
			&=0\quad_{\blacksquare}
		\end{align*}
\end{enumerate}
  
\end{enumerate}