\setcounter{chapter}{9}
\chapter{Processos Vetoriais Estacionários em Covariância}

\begin{enumerate}
	\item[\fbox{10.1}]
	
	Equação 10.2.19: $$\text{vec}\begin{bmatrix}
	\Gamma_0&\Gamma_1&\cdots&\Gamma_{p-1}\\
	\Gamma_1'&\Gamma_0&\cdots&\Gamma_{p-2}\\
	\Gamma_{p-1}'&\Gamma_{p-2}'&\cdots&\Gamma_0
	\end{bmatrix}=[\mathbf{I}_{r^2}-\mathbf{\mathscr{A}}]^{-1}\text{vec}(\mathbf{Q}).$$
	Sendo que $r=np$ e $$\mathbf{\mathscr{A}}\equiv \mathbf{F}_{(r\times r)}\otimes\mathbf{F}_{(r\times r)}$$
	
	
	Um processo escalar ($n=1$) $AR(p)$ tem o forma:
	
	\begin{align*}
		y_t &= \phi_1y_{t-1}+\phi_2y_{t-2}+\cdots+\phi_py_{t-p}+\varepsilon_t\\
	\end{align*}

Ou em forma vetorial:
	\begin{align*}
		\boldsymbol{\xi}_t = \mathbf{F}\boldsymbol{\xi}_{t-1}+\mathbf{v}_t,
	\end{align*}
em que $$\boldsymbol{\xi}_t=\begin{bmatrix}
y_t-\mu\\
y_{t-1}-\mu\\
\vdots\\
y_{t-p+1}-\mu
\end{bmatrix} ,\quad \mathbf{v}_t=\begin{bmatrix}
\varepsilon_t\\
0\\
\vdots\\
0
\end{bmatrix},$$
e conforme a equação 1.2.3: $$\mathbf{F}=\begin{bmatrix}
\phi_1 &\phi_2&\cdots&\phi_{p-1}&\phi_p\\
1&0&\cdots&0&0\\
0&1&\cdots&0&0\\
\vdots&\vdots&\cdots&\vdots&\vdots\\
0&0&\cdots&1&0
\end{bmatrix}$$


temos que 
$$\mathbf{F}_{(p\times p)}\otimes\mathbf{F}_{(p\times p)}=\begin{bmatrix}
\phi_1\mathbf{F}_{(p\times p)} &\phi_2\mathbf{F}_{(p\times p)}&\cdots&\phi_{p-1}\mathbf{F}_{(p\times p)}&\phi_p\mathbf{F}_{(p\times p)}\\
\mathbf{F}_{(p\times p)}&\mathbf{0}&\cdots&\mathbf{0}&\mathbf{0}\\
\mathbf{0}&\mathbf{F}_{(p\times p)}&\cdots&\mathbf{0}&\mathbf{0}\\
\vdots&\vdots&\cdots&\vdots&\vdots\\
\mathbf{0}&\mathbf{0}&\cdots&\mathbf{F}_{(p\times p)}&\mathbf{0}
\end{bmatrix}$$
\begin{landscape}
Então 
\begin{align*}
\mathbf{I}_{p^2}-(\mathbf{F}_{(p\times p)}\otimes\mathbf{F}_{(p\times p)})=\begin{bmatrix}
\mathbf{I}_p-\phi_1\mathbf{F}_{(p\times p)} &-\phi_2\mathbf{F}_{(p\times p)}&\cdots&-\phi_{p-1}\mathbf{F}_{(p\times p)}&-\phi_p\mathbf{F}_{(p\times p)}\\
-\mathbf{F}_{(p\times p)}&\mathbf{I}_p&\cdots&\mathbf{0}&\mathbf{0}\\
\mathbf{0}&-\mathbf{F}_{(p\times p)}&\cdots&\mathbf{0}&\mathbf{0}\\
\vdots&\vdots&\cdots&\vdots&\vdots\\
\mathbf{0}&\mathbf{0}&\cdots&-\mathbf{F}_{(p\times p)}&\mathbf{I}_p
\end{bmatrix}
\end{align*}

Reduzindo na forma escalonada temos


\begin{align*}
	\begin{bmatrix}
		(\mathbf{I}_p-\phi_p\mathbf{F}^{p}_{(p\times p)}-\phi_{p-1}\mathbf{F}^{p-1}_{(p\times p)}-\cdots-\phi_2\mathbf{F}^2_{(p\times p)}-\phi_1\mathbf{F}_{(p\times p)}) &(-\phi_p\mathbf{F}^{p-1}_{(p\times p)}-\phi_{p-1}\mathbf{F}^{p-2}_{(p\times p)}-\cdots-\phi_2\mathbf{F}_{(p\times p)})&\cdots&(-\phi_p\mathbf{F}^2_{(p\times p)}-\phi_{p-1}\mathbf{F}_{(p\times p)})&(-\phi_p\mathbf{F}_{(p\times p)})\\
		\mathbf{0}&\mathbf{I}_p&\cdots&\mathbf{0}&\mathbf{0}\\
		\mathbf{0}&\mathbf{0}&\cdots&\mathbf{0}&\mathbf{0}\\
		\vdots&\vdots&\cdots&\vdots&\vdots\\
		\mathbf{0}&\mathbf{0}&\cdots&\mathbf{0}&\mathbf{I}_p
	\end{bmatrix}
\end{align*}
Continua... não sei se estou no caminho certo!
\end{landscape}

		
	
\end{enumerate}