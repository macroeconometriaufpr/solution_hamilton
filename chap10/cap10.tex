\setcounter{chapter}{9}
\chapter{Processos Vetoriais Estacionários em Covariância}

\begin{enumerate}
	\item[\fbox{10.1}]
	
	Equação 10.2.19: $$\text{vec}\begin{bmatrix}
	\Gamma_0&\Gamma_1&\cdots&\Gamma_{p-1}\\
	\Gamma_1'&\Gamma_0&\cdots&\Gamma_{p-2}\\
	\vdots&\vdots&\vdots&\vdots\\
	\Gamma_{p-1}'&\Gamma_{p-2}'&\cdots&\Gamma_0
	\end{bmatrix}=[\mathbf{I}_{r^2}-\mathbf{\mathscr{A}}]^{-1}\text{vec}(\mathbf{Q}).$$
	Sendo que $r=np$ e $$\mathbf{\mathscr{A}}\equiv \mathbf{F}_{(r\times r)}\otimes\mathbf{F}_{(r\times r)}$$
	
	
	Um processo escalar ($n=1$) $AR(p)$ tem o forma:
	
	\begin{align*}
		y_t &= \phi_1y_{t-1}+\phi_2y_{t-2}+\cdots+\phi_py_{t-p}+\varepsilon_t\\
	\end{align*}

Ou em forma vetorial:
	\begin{align*}
		\boldsymbol{\xi}_t = \mathbf{F}\boldsymbol{\xi}_{t-1}+\mathbf{v}_t,
	\end{align*}

Avaliando a fórmula 10.2.19 para um processo escalar temos:

	$$\text{vec}\begin{bmatrix}
\gamma_0&\gamma_1&\cdots&\gamma_{p-1}\\
\gamma_1 &\gamma_0&\cdots&\gamma_{p-2}\\
\vdots&\vdots&\vdots&\vdots\\
\gamma_{p-1} &\gamma_{p-2} &\cdots&\gamma_0
\end{bmatrix}=[\mathbf{I}_{p^2}-\mathbf{\mathscr{A}}]^{-1}\text{vec}(\mathbf{Q}).$$

Com $$\mathbf{\mathscr{A}}\equiv \mathbf{F}_{(p\times p)}\otimes\mathbf{F}_{(p\times p)}$$

Temos ainda que $\mathbf{Q}$ para um processo escalar se torna:

\begin{align*}\mathbf{Q}=E[\mathbf{vv}']&=E\begin{Bmatrix}
\begin{bmatrix}
\varepsilon_t\\
0\\
\vdots\\
0
\end{bmatrix}&
\begin{bmatrix}
\varepsilon_t&0&\cdots&0
\end{bmatrix}\\
\end{Bmatrix}\\
&=E\begin{bmatrix}
	\varepsilon_t^2&0&\cdots&0\\
	0&0&\cdots&0\\
	\vdots&\vdots&\vdots&\vdots\\
	0&0&\cdots&0
\end{bmatrix}=\sigma^2\begin{bmatrix}
1&0&\cdots&0\\
0&0&\cdots&0\\
\vdots&\vdots&\vdots&\vdots\\
0&0&\cdots&0
\end{bmatrix}
	\end{align*}	
	Realizando a operação vec nos dois lados da equação  10.2.19 para o processo escalar:
	$$\begin{bmatrix}
	\gamma_0\\
	\gamma_1\\
	\vdots\\
	\gamma_{p-1}\\
	\gamma_1\\
	\gamma_0\\
	\vdots\\
	\gamma_{p-2}\\
	\vdots\\
	\gamma_{p-1}\\
	\gamma_{p-2}\\
	\vdots\\
	\gamma_0
	\end{bmatrix}=[\mathbf{I}_{p^2}-\mathbf{\mathscr{A}}]^{-1}\sigma^2\begin{bmatrix}
	1\\
	0\\
	\vdots\\
	0\\
	0\\
	0\\
	\vdots\\
	0\\
	\vdots\\
	0\\
	0\\
	\vdots\\
	0
	\end{bmatrix}$$
	
	dados que $[\mathbf{I}_{p^2}-\mathbf{\mathscr{A}}]^{-1}$ é uma matriz $p^2\times p^2$ e o vetor coluna do lado direito tem dimensão $P^2 \times 1$, o produto dos dois termos pode ser feito da seguinte forma. Deixe que $[\mathbf{I}_{p^2}-\mathbf{\mathscr{A}}]^{-1}$ seja igual a uma matriz $\mathbf{H}$  e que ela seja particionada em $p^2$ vetores coluna $\mathbf{h}_j$ de dimensão $p^2\times 1$ com $j=1,2,\cdots,p^2$.
	
	$$\mathbf{H}=\begin{bmatrix}
	\mathbf{h}_1&\mathbf{h}_2&\cdots&\mathbf{h}_{p^2}
	\end{bmatrix}$$
	
	Então o produto fica um vetor coluna de dimensão $p^2 \times 1$:
	$$\begin{bmatrix}
	\mathbf{h}_1&\mathbf{h}_2&\cdots&\mathbf{h}_{p^2}
	\end{bmatrix}\begin{bmatrix}
	1\\
	0\\
	\vdots\\
	o
	\end{bmatrix}_{(p^2\times 1)}=\begin{bmatrix}
	\mathbf{h}_1(1)+\mathbf{h}_2(0)+\cdots+\mathbf{h}_{p^2}(0)
	\end{bmatrix}_{(p^2\times 1)}=\begin{bmatrix}
	\mathbf{h}_1
	\end{bmatrix}_{(p^2\times 1)}$$
	
	Com isso os $p$ primeiros termos do lado esquerdo do processo escalar empilhado é o vetor coluna $$\begin{bmatrix}
	\gamma_0\\
	\gamma_1\\
	\gamma_2\\
	\vdots\\
	\gamma_{p-1}
	\end{bmatrix}$$
 e os $p$ primeiros termos do lado direito desta mesma equação é um vetor composto pelos $p$ primeiros elementos do vetor $$\sigma^2\begin{bmatrix}
 \mathbf{h}_1
 \end{bmatrix}_{(p^2\times 1)},$$
 que é equivalente aos $p$ primeiros elementos da primeira coluna de $$\sigma^2\mathbf{H}\equiv\sigma^2[\mathbf{I}_{p^2}-\mathbf{\mathscr{A}}]^{-1}\quad_{\blacksquare}$$.
\end{enumerate}