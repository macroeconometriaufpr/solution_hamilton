\chapter{Estimação de Máxima Verossimilhança}


\begin{enumerate}
	\item[\fbox{5.1}]
	
	
	Equação 5.4.16: $$\mathscr{L}(\boldsymbol{\theta})=\log f_{\mathbf{Y}}(\mathbf{y};\boldsymbol{\theta})=-\frac{T}{2}\log (2\pi)-\frac{1}{2}\sum \limits_{t=1}^T\log (d_{tt})-\frac{1}{2}\sum \limits _{t=1}^T\frac{\bar{y}_t^2}{d_{tt}}.$$
	
	os termos que possuem $\theta$ e $\sigma^2$ são as equações 5.4.11
	$$\bar{y}=y_t-\mu-\frac{\theta[1+\theta^2+\theta^4+\cdots+\theta^{2(t-2)}]}{[1+\theta^2+\theta^4+\cdots+\theta^{2(t-1)}]}\bar{y}_{t-1},$$
	e 5.4.12
	$$d_{tt}=E(\bar{Y}_t^2)=\sigma^2\frac{1+\theta^2+\theta^4+\cdots+\theta^{2t}}{1+\theta^2+\theta^4+\cdots+\theta^{2(t-1)}}.$$
	
	as duas equações são compostas por uma razão de somas finitas, com isso temos:
	
	\begin{align*}
		\frac{\theta[1+\theta^2+\theta^4+\cdots+\theta^{2(t-2)}]}{[1+\theta^2+\theta^4+\cdots+\theta^{2(t-1)}]}&=\frac{\theta[\frac{1-\theta^{2(t-2)}}{1-\theta^2}]}{\frac{1-\theta^{2(t-1)}}{1-\theta^2}}\\
		&=\theta\frac{[1-\theta^{2(t-2)}]}{1-\theta^{2(t-1)}}\\
	\end{align*}
	e
	\begin{align*}
		\frac{1+\theta^2+\theta^4+\cdots+\theta^{2t}}{1+\theta^2+\theta^4+\cdots+\theta^{2(t-1)}}&=\frac{\frac{1-\theta^{2t}}{1-\theta^2}}{\frac{1-\theta^{2(t-1)}}{1-\theta^2}}\\
		&=\frac{1-\theta^{2t}}{1-\theta^{2(t-1)}}
	\end{align*}
	
	Se $\sigma^2=\bar{\sigma}^2$ e $\theta=\bar{\theta}$, então:
	
	
	\begin{align*}
	\bar{y}&=y_t-\mu-\frac{\bar{\theta}[1+\bar{\theta}^2+\bar{\theta}^4+\cdots+\bar{\theta}^{2(t-2)}]}{[1+\bar{\theta}^2+\bar{\theta}^4+\cdots+\bar{\theta}^{2(t-1)}]}\bar{y}_{t-1},\\
	&=y_t-\mu-\bar{\theta}\frac{[1-\bar{\theta}^{2(t-2)}]}{1-\bar{\theta}^{2(t-1)}}\bar{y}_{t-1}\\
	\end{align*}
	\begin{align*}
	d_{tt}&=E(\bar{Y}_t^2)=\bar{\sigma}^2\frac{1+\bar{\theta}^2+\bar{\theta}^4+\cdots+\bar{\theta}^{2t}}{1+\bar{\theta}^2+\bar{\theta}^4+\cdots+\bar{\theta}^{2(t-1)}}\\
	&=\bar{\sigma}^2\frac{1-\bar{\theta}^{2t}}{1-\bar{\theta}^{2(t-1)}}
	\end{align*}
	
	Agora se $\sigma^2=\bar{\theta}^2\bar{\sigma}^2$ e $\theta=\bar{\theta}^{-1}$:
	
		\begin{align*}
	\bar{y}&=y_t-\mu-\frac{\bar{\theta}^{-1}[1+\bar{\theta}^{-2}+\bar{\theta}^{-4}+\cdots+\bar{\theta}^{-2(t-2)}]}{[1+\bar{\theta}^{-2}+\bar{\theta}^{-4}+\cdots+\bar{\theta}^{-2(t-1)}]}\bar{y}_{t-1},\\
	&=y_t-\mu-\bar{\theta}^{-1}\frac{[1-\bar{\theta}^{-2(t-2)}]}{1-\bar{\theta}^{-2(t-1)}}\bar{y}_{t-1}\\
	\end{align*}
	\begin{align*}
	d_{tt}&=E(\bar{Y}_t^2)=\bar{\theta}\bar{\sigma}^2\frac{1+\bar{\theta}^{-2}+\bar{\theta}^{-4}+\cdots+\bar{\theta}^{-2t}}{1+\bar{\theta}^{-2}+\bar{\theta}^{-4}+\cdots+\bar{\theta}^{-2(t-1)}}\\
	&=\bar{\theta}^2\bar{\sigma}^2\frac{1-\bar{\theta}^{-2t}}{1-\bar{\theta}^{-2(t-1)}}
	\end{align*}
	
	As duas formas são equivalentes pois
	
	$$\bar{\theta}^{-1}\frac{[1-\bar{\theta}^{-2(t-2)}]}{1-\bar{\theta}^{-2(t-1)}}\equiv \bar{\theta}\frac{[1-\bar{\theta}^{2(t-2)}]}{1-\bar{\theta}^{2(t-1)}}$$
	e
	$$\bar{\theta}^2\frac{1-\bar{\theta}^{-2t}}{1-\bar{\theta}^{-2(t-1)}}\equiv \frac{1-\bar{\theta}^{2t}}{1-\bar{\theta}^{2(t-1)}}\;\;_\blacksquare$$
	
	
	\item[\fbox{5.2}]
	
	
		Equação 5.7.6: $$\mathscr{L}(\boldsymbol{\theta}) = -1.5\theta_1^2-2\theta_2^2$$
		Equação 5.7.12: $$\boldsymbol{\theta}^{(1)}-\boldsymbol{\theta}^{(0)}=[\mathbf{H}(\boldsymbol{\theta}^{(0)})]^{-1}\mathbf{g}(\boldsymbol{\theta}^{(0)}).$$
		Dado que $\boldsymbol{\theta}^{(0)}=[-1,1]'$.
		
		\begin{align*}
			\mathbf{g}(\boldsymbol{\theta})&=\begin{bmatrix}
				-3\theta_1\\
				-4\theta_2
			\end{bmatrix}\Rightarrow
		\mathbf{g}(\boldsymbol{\theta}^{(0)})=\begin{bmatrix*}[r]
			3\\
			-4			
		\end{bmatrix*}\\
			\mathbf{H}(\boldsymbol{\theta}^{(0)})&=\begin{bmatrix*}[r]
				3&0\\
				0&4
			\end{bmatrix*}\Rightarrow
				[\mathbf{H}(\boldsymbol{\theta}^{(0)})]^{-1}=\frac{1}{12}
				\begin{bmatrix*}[r]
					4&0\\
					0&3
				\end{bmatrix*}\\
			\\
			\boldsymbol{\theta}^{(1)}-\boldsymbol{\theta}^{(0)}&=[\mathbf{H}(\boldsymbol{\theta}^{(0)})]^{-1}\mathbf{g}(\boldsymbol{\theta}^{(0)})\Rightarrow\\
			\boldsymbol{\theta}^{(1)}&=\begin{bmatrix*}[r]
				-1\\
				1
			\end{bmatrix*}+\begin{bmatrix*}[r]
			1/3&0\\
			0&1/4
		\end{bmatrix*}
	\begin{bmatrix*}[r]
		3\\
		-4			
	\end{bmatrix*}\\
	&=\begin{bmatrix*}[r]
		-1\\
		1
	\end{bmatrix*}+\begin{bmatrix*}[r]
	1\\
	-1
\end{bmatrix*}\\
&=\begin{bmatrix*}[r]
	0\\
	0
\end{bmatrix*}\;\;_\blacksquare
		\end{align*}
	
	\item[\fbox{5.3}]
	
	
	\begin{enumerate}
		\item %5.3a
		
		\begin{align*}
			\mathscr{L}(\mu, \sigma^2;y_1,y_2,\cdots,y_T)&=-\frac{T}{2}\log(2\pi)-\frac{T}{2}\log(\sigma^2)-\sum\limits_{t=1}^{T}\frac{(y_t-\mu)^2}{2\sigma^2}\\
			\\
			\frac{\partial \mathscr{L}}{\partial\mu}&=0\Rightarrow \sum\limits_{t=1}^{T}\frac{2(y_t-\mu)}{2\sigma^2}=0,\\
			\sum\limits_{t=1}^{T}y_t&=T\mu\Rightarrow\hat{\mu}=T^{-1}\sum\limits_{t=1}^{T}y_t\\
			\\
			\frac{\partial \mathscr{L}}{\partial\sigma^2}&=0\Rightarrow-\frac{T}{2\sigma^2}-\bigg[-\sum\limits_{t=1}^{T}\frac{2(y_t-\mu)^2}{4\sigma^4}\bigg]=0,\\
			\frac{T}{2\sigma^2}&=\frac{1}{2\sigma^4}\sum\limits_{t=1}^{T}(y_t-\mu)^2\Rightarrow\hat{\sigma}^2=T^{-1}\sum\limits_{t=1}^{T}(y_t-\hat{\mu})^2\;\;_\blacksquare
		\end{align*}
	
		\item %5.3b
		
		\begin{align*}
			\frac{\partial^2\mathscr{L}}{\partial\mu^2}&=-\sum\limits_{t=1}^T\frac{2}{2\sigma^2}=-\frac{T}{\sigma^2}\\
			\\
			\frac{\partial^2\mathscr{L}}{\partial\sigma^2\partial\mu}&=-\sum\limits_{t=1}^T\frac{4(y_t-\mu)}{4\sigma^4}=-\sum\limits_{t=1}^T\frac{(y_t-\mu)}{\sigma^4}\\
			\\
			\frac{\partial^2\mathscr{L}}{\partial(\sigma^2)^2}&=\frac{2T}{4\sigma^4}-\sum\limits_{t=1}^T\frac{4\sigma^2(y_t-\mu)^2}{4\sigma^8}=\frac{T}{2\sigma^4}-\sum\limits_{t=1}^T\frac{(y_t-\mu)^2}{\sigma^6}=\\
			\\
			\frac{\partial^2\mathscr{L}}{\partial\mu\partial\sigma^2}&=-\sum\limits_{t=1}^{T}\frac{4(y_t-\mu)}{4\sigma^4}=-\sum\limits_{t=1}^{T}\frac{(y_t-\mu)}{\sigma^4}
		\end{align*}
		Então a matriz Hessiana fica:
		
		\begin{align*}
			\frac{\partial^2 \mathscr{L}(\boldsymbol{\theta})}{\partial\boldsymbol{\theta}\partial\boldsymbol{\theta}'}=\begin{bmatrix}
				-\frac{T}{\sigma^2}&-\sum\limits_{t=1}^T\frac{(y_t-\mu)}{\sigma^4}\\
				-\sum\limits_{t=1}^T\frac{(y_t-\mu)}{\sigma^4}&\frac{T}{2\sigma^4}-\sum\limits_{t=1}^T\frac{(y_t-\mu)^2}{\sigma^6}
			\end{bmatrix}
		\end{align*}
		Se avaliada em $\theta=\hat{\theta}$:
		
			
		\begin{align*}
			\mathscr{\hat{I}}_{2D}=-T^{-1}\frac{\partial^2 \mathscr{L}(\boldsymbol{\theta})}{\partial\boldsymbol{\theta}\partial\boldsymbol{\theta}'}\Bigg|_{\boldsymbol{\theta}=\hat{\boldsymbol{\theta}}}=E\begin{Bmatrix}-T^{-1}\begin{bmatrix}
				-\frac{T}{\sigma^2}&-\sum\limits_{t=1}^T\frac{(y_t-\mu)}{\sigma^4}\\
				-\sum\limits_{t=1}^T\frac{(y_t-\mu)}{\sigma^4}&\frac{T}{2\sigma^4}-\sum\limits_{t=1}^T\frac{(y_t-\mu)^2}{\sigma^6}
			\end{bmatrix}\end{Bmatrix}
		\end{align*}
	\pagebreak
	
	Dado que $E[y_t]=\mu$ e $E[(y_t-\mu)^2]=\sigma_2$:
	
			$$\mathscr{\hat{I}}_{2D}=\begin{bmatrix}
			\frac{1}{\sigma^2}&0\\
			0&\frac{1}{2\sigma^4}
			\end{bmatrix}$$
			
	\item %5.3c
	
	A inversa de  $\mathscr{\hat{I}}_{2D}$ é:
	
	\begin{align*}
		\mathscr{\hat{I}}_{2D}^{-1}={2\sigma^6}\begin{bmatrix}
		\frac{1}{2\sigma^4}	&0\\
			0&\frac{1}{\sigma^2}\end{bmatrix}
		=\begin{bmatrix}
			\sigma^2	&0\\
			0&2\sigma^4\end{bmatrix}
	\end{align*}

	Multiplicando por $T^{-1}$:
	
	$$T^{-1}\mathscr{\hat{I}}_{2D}^{-1}=\begin{bmatrix}
	\sigma^2/T	&0\\
	0&2\sigma^4/T\end{bmatrix}$$
	
	com isso: 
	$$\begin{bmatrix}
	\hat{\mu}\\
	\hat{\sigma}^2
	\end{bmatrix}\approx N \begin{pmatrix}
	\begin{bmatrix}
	\mu\\
	\sigma^2
	\end{bmatrix},
	\begin{bmatrix}
	\sigma^2/T	&0\\
	0&2\sigma^4/T\end{bmatrix}
	\end{pmatrix}_{\;\;\blacksquare}$$
	\end{enumerate}
	
\end{enumerate}