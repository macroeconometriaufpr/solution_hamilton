\chapter{Previsão}

\begin{enumerate}
	\item[\fbox{4.1}]
	
	Fórmula 4.3.6: 
	
	\begin{align*}
		\boldsymbol{\alpha}^{(m)}{'}&\equiv
		\begin{bmatrix}
\alpha_0^{m}&\alpha_1^m&\alpha_2^m&\cdots&\alpha_m^m
		\end{bmatrix}\\
		&=\begin{bmatrix}
		\mu&\gamma_1+\mu^2&\gamma_2+\mu^2&\cdots&\gamma^m+\mu^2
		\end{bmatrix}\\
		&\times\begin{bmatrix}
		1&\mu&\mu&\cdots&\mu\\[0.3cm]
		\mu&\gamma_0+\mu^2&\gamma_1+\mu^2&\cdots&\gamma_{m-1}+\mu^2\\[0.3cm]
		\mu&\gamma_1+\mu^2&\gamma_0+\mu^2&\cdots&\gamma_{m-2}+\mu^2\\[0.3cm]
		\vdots&\vdots&\vdots&\ddots&\vdots\\[0.3cm]
		\mu&\gamma{m-1}+\mu^2&\gamma_{m-2}+\mu^2&\cdots&\gamma_0+\mu^2
		\end{bmatrix}^{-1}\\
		\\
		&\equiv  {E}\big[Y_{t+1}X_t'\big] {E}\big[X_tX_t'\big]^{-1}\\
		\text{Para }X_t=\begin{bmatrix}
		1&Y_t
		\end{bmatrix}'\\
		&=\begin{bmatrix}
		 {E}[Y_{t+1}]&
		 {E}[Y_{t+1}Y_t]
		\end{bmatrix}
		\times
		\begin{bmatrix}
		1&E[Y_{t}]\\[0.3cm]
		E[Y_{t}]&E[Y_{t}^2]
		\end{bmatrix}^{-1}\\
		&=\begin{bmatrix}
		\mu&
		\gamma_1+\mu^2
		\end{bmatrix}
		\times
		\frac{1}{\gamma_0}
		\begin{bmatrix}
		\gamma_0+\mu^2&-\mu\\
		-\mu&1
		\end{bmatrix}\\
		\boldsymbol{\alpha}'&=
		\frac{1}{\gamma_0}
		\begin{bmatrix}
		\mu\gamma_0+\mu^3-\mu\gamma_1-\mu^3&-\mu^2+\gamma_1+\mu^2
		\end{bmatrix}\\
		&=\frac{1}{\gamma_0}
		\begin{bmatrix}
		\mu\gamma_0-\mu\gamma_1&\gamma_1
		\end{bmatrix}\\
		&=\begin{bmatrix}
		(1-\rho_1)\mu&\rho_1
		\end{bmatrix}
	\end{align*}
	\begin{align*}
	\hat{E}[Y_{t+1}|Y_t]&=\boldsymbol{\alpha}'X_t\\
&=\begin{bmatrix}
(1-\rho_1)\mu&\rho_1
\end{bmatrix}
	\begin{bmatrix}
	1\\
	Y_t
	\end{bmatrix}\\
	\hat {E}\big[Y_{t+1}|Y_t\big]&=(1-\rho_1)\mu+\rho_1Y_t
	\end{align*}
	Portanto a previsão para o período $t+1$ com base em informações somente do período $t$ é uma média ponderada entre o valor esperado de $Y_t$, $\mu$ e $Y_t$, em que o fator de ponderação é o coeficiente de autocorrelação de primeira defasagem, $\rho_1$.
\end{enumerate}